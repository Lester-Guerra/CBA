\documentclass[letterpaper, twoside]{article}
\usepackage[english]{babel}
\usepackage{graphicx}
\usepackage{anysize}
\usepackage{hyperref}
\usepackage{fontspec}
\setmainfont{Montserrat}
\usepackage{multirow}
\usepackage{array}
\usepackage{enumitem}
\setlength{\headheight}{14.5pt}
\usepackage{geometry}
\usepackage{fancyhdr}
\pagestyle{fancy}
\usepackage{tikz}
\usetikzlibrary{calc}
\usepackage{tikzpagenodes}
\usepackage{xcolor}
\definecolor{AzulTexto}{RGB}{0, 32, 96}

\renewcommand{\headrulewidth}{0pt}

\usepackage{setspace}

\fancyhf{} % Pone en blanco el encabezado y pie de página

\fancyhead[E]{
  \begin{tikzpicture}[remember picture,overlay]
    \node[
      inner sep=0pt,outer sep=0pt,anchor=north west
    ] at (current page.north west) {
      \includegraphics[width=\paperwidth]{topeven.pdf}
    };
  \end{tikzpicture}
}
\fancyhead[O]{
  \begin{tikzpicture}[remember picture,overlay]
    \node[
      inner sep=0pt,outer sep=0pt,anchor=north west
    ] at (current page.north west) {
      \includegraphics[width=\paperwidth]{topodd.pdf}
    };
  \end{tikzpicture}
}

\fancyfoot[C]{\ \includegraphics{Pie_de_pagina.jpg}\ }

% MÁRGENES
\marginsize{85pt}%izquierda
{85pt} %derecha
{105pt} %arriba
{40pt} %abajo

% Sin sangría
\setlength{\parindent}{0pt}

%Esto evita que las palabras se dividan con guión.
\tolerance=1
\emergencystretch=\maxdimen
\hyphenpenalty=10000
\hbadness=10000

\input{macros}

\begin{document}
\setlength{\footskip}{49pt}
\color{AzulTexto}

  \begin{tikzpicture}[remember picture,overlay]
    \node[
      yshift=-60pt, xshift=10pt,anchor=north west
    ] at (current page.north west) {
      \includegraphics{Comunicado_De_Prensa.jpg}
    };
    \node[
      yshift=-98pt, xshift=20pt,anchor=north west
    ] at (current page.north west) {
      \footnotesize No. CBA-\nMes-\anio
    };
  \end{tikzpicture}

\;
\\[\baselineskip]
\begin{center}
{\bf\large El INE da a conocer el costo mensual per cápita a \mes\ del \anio \\
  de la Canasta Básica Alimentaria Urbana y Rural}
\end{center}
\;
\\[\baselineskip]
Guatemala, \diapub\ de \mespub\ de \aniopub. El Instituto Nacional de Estadística
(INE), presenta resultados de la Canasta Básica Alimentaria Urbana (CBAU) y
Canasta Básica Alimentaria Rural (CBAR), además la Canasta Ampliada Urbana (CAU)
y Canasta Ampliada Rural (CAR) al mes de \mes\ del \anio.
\\[\baselineskip]
La CBAU y la CBAR implementada a partir de enero del 2024, está integrada por 14
grupos alimenticios, excluyendo el 12 que identifica Bebidas alcohólicas.
El costo de adquisición per cápita mensual de la CBAU al mes de \mes\ del \anio\
fue de Q \CBAU\ y para la CBAR fue de Q \CBAR\ conforme a la metodología que se
indica, su estructura, costo por grupo alimenticio y total, se muestran en la
página siguiente en los cuadros 1 y 2 respectivamente.

\setcounter{section}{1}
\subsection{Acerca de la CBA y CA}

La Canasta Básica Alimentaria (CBA) es el "Mínimo alimentario conformado por un
conjunto de alimentos básicos, en cantidades apropiadas y suficientes para
satisfacer por lo menos las necesidades energéticas y proteínicas de la familia
u hogar de referencia"(Menchú y Osegueda, 2002).
\\[\baselineskip]
La CBAU contiene 66 productos y la CBAR 60, cuantifican los gramos sugeridos
para una persona promedio y que cubren el requerimiento energético de 2,052
calorías para la CBAU y 2,172 calorías para la CBAR.
En su formulación, la CBA sigue la metodología de gasto que observa los patrones
de consumo efectivo, es decir no es una canasta con fines normativos, como
ocurriera si fuese para fines dietéticos.

\newpage

\begin{tabular}{l l}
  Cuadro 1 & Canasta Básica Alimentaria \\
  & Costo mensual per cápita por grupo alimenticio Urbana* \\
  & \Mes\ \anio
\end{tabular}
\\[\baselineskip]
{\renewcommand{\arraystretch}{1.5}
\color{black} \begin{tabular}{|c|l|r|r|r|r|c|}
  \hline
  \multirow{2}{4mm}{\bf No.} & \multirow{2}{48mm}{\bf Grupo Alimenticio} &
    \multirow{2}{16mm}{\bf Cantidad gramos mes} &
    \multirow{2}{17mm}{\bf Costo mensual \mesant} &
    \multirow{2}{17mm}{\bf Costo mensual \mes} &
    \multicolumn{1}{c|}{\bf Variación} & \\
  & & & & & \multicolumn{1}{c|}{\bf\%} & \\ \hline
  \GruposComU
  \multicolumn{3}{|c|}{\bf Costo per cápita mensual} & \bf\CBAUant & \bf\CBAU &
    \bf\VarU & \SignoU \\
\hline
\end{tabular}}
{\scriptsize
\\[\baselineskip]
Fuente: Elaboración INE\\
{*} El sistema calcula los valores a 16 cifras decimales. Se aproximan a dos
decimales.
}

\newpage

\begin{tabular}{l l}
  Cuadro 2 & Canasta Básica Alimentaria \\
  & Costo mensual per cápita por grupo alimenticio Rural* \\
  & \Mes\ \anio
\end{tabular}
\\[\baselineskip]
{\renewcommand{\arraystretch}{1.5}
\color{black} \begin{tabular}{|c|l|r|r|r|r|c|}
  \hline
  \multirow{2}{4mm}{\bf No.} & \multirow{2}{48mm}{\bf Grupo Alimenticio} &
    \multirow{2}{16mm}{\bf Cantidad gramos mes} &
    \multirow{2}{17mm}{\bf Costo mensual \mesant} &
    \multirow{2}{17mm}{\bf Costo mensual \mes} &
    \multicolumn{1}{c|}{\bf Variación} & \\
  & & & & & \multicolumn{1}{c|}{\bf\%} & \\ \hline
  \GruposComR
  \multicolumn{3}{|c|}{\bf Costo per cápita mensual} & \bf\CBARant & \bf\CBAR &
    \bf\VarR & \SignoR \\
\hline
\end{tabular}}
{\scriptsize
\\[\baselineskip]
Fuente: Elaboración INE\\
{*} El sistema calcula los valores a 16 cifras decimales. Se aproximan a dos
decimales.
}

\newpage

La Canasta Ampliada (CA) se define como el conjunto de bienes y servicios que
satisfacen las necesidades ampliadas de los miembros de un hogar y conforme los
datos declarados por los hogares, incluye alimentación, bebidas alcohólicas,
ropa y calzado, vivienda, mobiliario, salud, transporte, comunicaciones,
recreación y cultura, educación, restaurantes y hoteles, servicios financieros y
cuidado personal. 
\\[\baselineskip]
El costo de adquisición de la CAU per cápita mensual a \mes\ del \anio\ fue de
Q.\CAU\ y el costo de la CAR fue de Q.\CAR, en su cálculo se ha empleado el
coeficiente de Orshansky de 2.421 para la CAU, y 1.968 para la CAR, resultado de
la ENIGH 2022-2023.
\\[\baselineskip]
Con esta publicación el INE continúa su esfuerzo por brindar información
oportuna y de acceso a la población, modernizando el Sistema Integral de Precios
(SIP) que este gestiona. 
\\[\baselineskip]
Los documentos base y otras series históricas pueden ser consultados en el sitio
de internet del INE: \href{http://www.ine.gob.gt}{www.ine.gob.gt}.
\end{document}