\documentclass[letterpaper, 12pt, twoside]{article}
\usepackage[spanish]{babel}
\usepackage{graphicx}
\usepackage{anysize}
\usepackage{hyperref}
\usepackage{fontspec}
\setmainfont{Arial Narrow}
\usepackage{pdfpages}
\usepackage{multirow}
\usepackage{longtable}
\usepackage{array}
\usepackage{enumitem}
\setlength{\headheight}{14.5pt}
\usepackage{geometry}
\usepackage{fancyhdr}
\pagestyle{fancy}
\usepackage{tikz}
\usetikzlibrary{calc}
\usepackage{tikzpagenodes}
\usepackage{xcolor}
\definecolor{AzulTexto}{RGB}{0, 32, 96}
\definecolor{AzulAcento}{RGB}{0, 167, 229}
\definecolor{Gris}{RGB}{128, 128, 128}

\renewcommand{\headrulewidth}{0pt}
\renewcommand{\thesubsection}{\arabic{subsection}}

\usepackage{setspace}

\fancyhf{} % Pone en blanco el encabezado y pie de página

\fancyhead[E]{
  \begin{tikzpicture}[remember picture,overlay]
    \node[
      inner sep=0pt,outer sep=0pt,anchor=north west
    ] at (current page.north west) {
      \includegraphics[width=\paperwidth]{topeven.png}
    };
  \end{tikzpicture}
}
\fancyhead[O]{
  \begin{tikzpicture}[remember picture,overlay]
    \node[
      inner sep=0pt,outer sep=0pt,anchor=north west
    ] at (current page.north west) {
      \includegraphics[width=\paperwidth]{topodd.png}
    };
  \end{tikzpicture}
}

\fancyfoot[RO]{
  {\footnotesize
    {\color{AzulAcento} CANASTA BÁSICA ALIMENTARIA}
    {\color{Gris} | Actualización metodológica 2024}
  } \quad \thepage
}
\fancyfoot[LE]{
  \thepage \quad
  {\footnotesize
    {\color{AzulAcento} CANASTA BÁSICA ALIMENTARIA}
    {\color{Gris} | Actualización metodológica 2024}
  }
}

% MÁRGENES
\marginsize{80pt}%izquierda
{80pt} %derecha
{45pt} %arriba
{40pt} %abajo

% Sin sangría
\setlength{\parindent}{0pt}

%Esto evita que las palabras se dividan con guión.
\tolerance=1
\emergencystretch=\maxdimen
\hyphenpenalty=10000
\hbadness=10000

\input{macros}

\begin{document}
\includepdf[
  pagecommand={
    \thispagestyle{empty}
    \begin{tikzpicture}[remember picture, overlay]
      \node at ([yshift=1cm, xshift=-3cm] current page) {
        \color{white}{\Huge Canasta Básica Alimentaria}
      };
      \node at ([yshift=-1cm, xshift=-3cm] current page) {
        \color{white}{\Huge \textbf{\Mes\ \anio}}
      };
      \node at ([yshift=-12cm] current page) {
        \color{white}{\Large Guatemala, \mespub\ de \aniopub }
      };
    \end{tikzpicture}
  }
]{./portada.pdf}
\addtocounter{page}{-1}
\color{AzulTexto}
\;
\\[3\baselineskip]
\section*{Introducción}
\;
\\[3\baselineskip]
El Instituto Nacional de Estadística (INE), a través de este documento presenta
los resultados de la Canasta Básica Alimentaria Urbana (CBAU) y Canasta Básica
Alimentaria Rural (CBAR) y su costo de adquisición mensual per cápita, tanto
por grupo alimenticio como por producto.
\\[2\baselineskip]
El costo de adquisición de la CBAU per cápita mensual a \mes del \anio\ fue de
Q.\CBAU\ y de la CBAR fue de Q.\CBAR .
\\[\baselineskip]
El costo de adquisición de la Canasta Ampliada Urbana (CAU) per cápita mensual a
\mes\ del \anio\ fue de Q.\CAU\ y de la Canasta Ampliada Rural (CAR) fue de
Q.\CAR\, y en su cálculo se ha utilizado el coeficiente de Orshansky para el
área urbana (2.421) y para el área rural (1.968) según los resultados de la
ENIGH 2022-2023.
\\[\baselineskip]
Con esta publicación, el INE continúa su esfuerzo por brindar información
oportuna y de acceso a la población, modernizando el Sistema de Precios que este
gestiona.
Los documentos base y otras series históricas pueden ser consultados en el sitio
de internet del INE.
Cualquier comentario puede dirigirse a
\href{mailto:comunicacion@ine.gob.gt}{comunicacion@ine.gob.gt}.

\newpage

\;
\section*{Canasta Básica Alimentaria (CBA)}

La CBA es un conjunto de alimentos que constituyen un mínimo necesario para
satisfacer al menos las necesidades energéticas y proteínicas de una persona;
y que se ajustan al patrón cultural, capacidad adquisitiva y la disponibilidad y
precios de los alimentos a nivel local.
\\[2\baselineskip]
La actualización de la CBA representa un cambio significativo en comparación con
la última revisión realizada en el año 2020.
Entre los cambios más destacados se encuentra la metodología utilizada, que
ahora prioriza la obtención de la población de referencia a través de las
carencias.
Otro cambio importante implica la inclusión de los gastos relacionados con la
comida y bebida fuera del hogar en el cálculo del costo de la CBA.
Además, se modificó el método para determinar este costo, optando por precios
medianos en lugar de precios promedio.
También se segmentó la canasta en área urbana y rural.
\\[\baselineskip]
La CBAU contiene 66 productos y la CBAR 60, cuantifican los gramos sugeridos
para un hogar de 4.16 miembros que cubren el requerimiento energético de 2,052
calorías para la CBAU y 4.80 miembros y 2,172 calorías para la CBAR.
En su formulación, la CBA sigue la metodología de gasto que observa los patrones
de consumo efectivo, es decir no es una canasta con fines normativos, como
ocurriera si fuese para fines dietéticos.
\\[2\baselineskip]
La CBAU y la CBAR implementada a partir de enero del 2024, está integrada por 14
grupos alimenticios, excluyendo el 12 que identifica Bebidas alcohólicas.
Los 13 restantes identifican 66 productos alimenticios para la CBAU y 60 para la
CBAR.
Posteriormente, su estructura, costo por grupo alimenticio y total, se muestran
a continuación.

\newpage

{\begin{tabular}{l l}
  Cuadro 1 & Canasta Básica Alimentaria \\
  & Costo mensual per cápita por grupo alimenticio Urbana* \\
  & \Mes\ \anio
\end{tabular}
\\[\baselineskip]
\renewcommand{\arraystretch}{1.5}
{\color{black}\small \begin{tabular}{|c|l|r|r|r|r|r|}
  \hline
  \multirow{2}{4mm}{\bf No.} & \multirow{2}{48mm}{\bf Grupo Alimenticio} &
    \multirow{2}{17mm}{\bf Kilocalorías por persona} &
    \multirow{2}{16mm}{\bf Cantidad gramos mes} &
    \multirow{2}{17mm}{\bf Costo mensual \mesant} &
    \multirow{2}{17mm}{\bf Costo mensual \mes} &
    \multicolumn{1}{c|}{\bf Variación} \\
  & & & & & & \multicolumn{1}{c|}{\bf\%} \\ \hline
  \GruposU
  \multicolumn{4}{|c|}{\bf Costo per cápita mensual} & \bf\CBAUant & \bf\CBAU &
    \bf\VarU \\
\hline
\end{tabular}}}
{\scriptsize
\\[\baselineskip]
Fuente: Elaboración INE\\
{*} El sistema calcula los valores a 16 cifras decimales. Se aproximan a dos
decimales.
}

\newpage

\begin{tabular}{l l}
  Cuadro 2 & Canasta Básica Alimentaria \\
  & Costo mensual per cápita por producto Urbana* \\
  & \Mes\ \anio
\end{tabular}
\renewcommand{\arraystretch}{1.25}
{\scriptsize{\color{black} \begin{longtable}{|c|l|r|r|r|r|r|}
  \hline
  \multirow{3}{2mm}{\bf No.} & \multirow{3}{79mm}{\bf Producto} &
    \multirow{3}{12mm}{\bf Kilocalorías por persona} &
    \multirow{3}{9mm}{\bf Cantidad gramos mes} &
    \multirow{3}{11mm}{\bf Costo mensual \mesant} &
    \multirow{3}{11mm}{\bf Costo mensual \mes} &
    \multicolumn{1}{c|}{\bf Variación} \\
    & & & & & & \multicolumn{1}{c|}{\bf\%} \\
    & & & & & & \\ \hline
  \endfirsthead
  \hline
  \multirow{3}{2mm}{\bf No.} & \multirow{3}{79mm}{\bf Producto} &
    \multirow{3}{12mm}{\bf Kilocalorías por persona} &
    \multirow{3}{9mm}{\bf Cantidad gramos mes} &
    \multirow{3}{11mm}{\bf Costo mensual \mesant} &
    \multirow{3}{11mm}{\bf Costo mensual \mes} &
    \multicolumn{1}{c|}{\bf Variación} \\
    & & & & & & \multicolumn{1}{c|}{\bf\%} \\
    & & & & & & \\ \hline
  \endhead
  \ProdsU
  \multicolumn{4}{|c|}{\bf Costo per cápita mensual} & \bf\CBAUant & \bf\CBAU &
  \\ \hline
\end{longtable}
}\;
\begin{spacing}{1}
Fuente: Elaboración INE\\
{\tiny 1/} Se refiere a los productos definidos en la CBAU, para efectos de la
consignación de precios se han especificado algunos casos de la forma siguiente:
cereales de bolsa o caja solo se consideró las hojuelas de maíz simple, en
fideos se cotiza cabello de ángel, coditos, conchitas, corbatín y tornillos,
para las salchicha se cotiza solo las de pollo y cerdo, para el jamón se cotiza
la variedad de pollo, para las longanizas y chorizos se cotiza los artesanales,
para el cilantro, perejil y hierbabuena se cotiza solo cilantro y hierbabuena,
para los frijoles negros secos se cotiza solo la variedad frijoles negros secos
a granel, bases para sopa se cotiza solo la sopa con fideos o arroz, café
instantáneo se cotiza solo la variedad en frasco, agua purificada se cotiza la
variedad en bolsa o botella, para las gaseosas se cotiza las gaseosas familiares
(litro o más).
{\tiny 2/} La cantidad en gramos corresponde a la establecida en los estudios de
canasta de referencia.  En \mes, conforme la metodología indicada. \\
{*} El sistema calcula los valores a 16 cifras decimales. Se aproximan a dos
decimales.
\end{spacing}
}

\newpage

{\begin{tabular}{l l}
  Cuadro 3 & Canasta Básica Alimentaria \\
  & Costo mensual per cápita por grupo alimenticio Rural* \\
  & \Mes\ \anio
\end{tabular}
\\[\baselineskip]
\renewcommand{\arraystretch}{1.5}
{\color{black}\small \begin{tabular}{|c|l|r|r|r|r|r|}
  \hline
  \multirow{2}{4mm}{\bf No.} & \multirow{2}{48mm}{\bf Grupo Alimenticio} &
    \multirow{2}{17mm}{\bf Kilocalorías por persona} &
    \multirow{2}{16mm}{\bf Cantidad gramos mes} &
    \multirow{2}{17mm}{\bf Costo mensual \mesant} &
    \multirow{2}{17mm}{\bf Costo mensual \mes} &
    \multicolumn{1}{c|}{\bf Variación} \\
  & & & & & & \multicolumn{1}{c|}{\bf\%} \\ \hline
  \GruposR
  \multicolumn{4}{|c|}{\bf Costo per cápita mensual} & \bf\CBARant & \bf\CBAR &
    \bf\VarR \\
\hline
\end{tabular}}}
{\scriptsize
\\[\baselineskip]
Fuente: Elaboración INE\\
{*} El sistema calcula los valores a 16 cifras decimales. Se aproximan a dos
decimales.
}

\newpage

\begin{tabular}{l l}
  Cuadro 4 & Canasta Básica Alimentaria \\
  & Costo mensual per cápita por producto Rural* \\
  & \Mes\ \anio
\end{tabular}
\renewcommand{\arraystretch}{1.25}
{\scriptsize{\color{black} \begin{longtable}{|c|l|r|r|r|r|r|}
  \hline
  \multirow{3}{2mm}{\bf No.} & \multirow{3}{79mm}{\bf Producto} &
    \multirow{3}{12mm}{\bf Kilocalorías por persona} &
    \multirow{3}{9mm}{\bf Cantidad gramos mes} &
    \multirow{3}{11mm}{\bf Costo mensual \mesant} &
    \multirow{3}{11mm}{\bf Costo mensual \mes} &
    \multicolumn{1}{c|}{\bf Variación} \\
    & & & & & & \multicolumn{1}{c|}{\bf\%} \\
    & & & & & & \\ \hline
  \endfirsthead
  \hline
  \multirow{3}{2mm}{\bf No.} & \multirow{3}{79mm}{\bf Producto} &
    \multirow{3}{12mm}{\bf Kilocalorías por persona} &
    \multirow{3}{9mm}{\bf Cantidad gramos mes} &
    \multirow{3}{11mm}{\bf Costo mensual \mesant} &
    \multirow{3}{11mm}{\bf Costo mensual \mes} &
    \multicolumn{1}{c|}{\bf Variación} \\
    & & & & & & \multicolumn{1}{c|}{\bf\%} \\
    & & & & & & \\ \hline
  \endhead
  \ProdsR
  \multicolumn{4}{|c|}{\bf Costo per cápita mensual} & \bf\CBARant & \bf\CBAR &
  \\ \hline
\end{longtable}
}\;
\begin{spacing}{1}
Fuente: Elaboración INE\\
{\tiny 1/} Se refiere a los productos definidos en la CBAR, para efectos de la
consignación de precios se han especificado algunos casos de la forma siguiente:
cereales de bolsa o caja solo se consideró las hojuelas de maíz simple, en
fideos se cotiza cabello de ángel, coditos, conchitas, corbatín y tornillos,
para las salchicha se cotiza solo las de pollo y cerdo, para el jamón se cotiza
la variedad de pollo, para las longanizas y chorizos se cotiza los artesanales,
para el cilantro, perejil y hierbabuena se cotiza solo cilantro y hierbabuena,
para los frijoles negros secos se cotiza solo la variedad frijoles negros secos
a granel, bases para sopa se cotiza solo la sopa con fideos o arroz, café
instantáneo se cotiza solo la variedad en frasco, agua purificada se cotiza la
variedad en bolsa o botella, para las gaseosas se cotiza las gaseosas familiares
(litro o más).
{\tiny 2/} La cantidad en gramos corresponde a la establecida en los estudios de
canasta de referencia.  En \mes, conforme la metodología indicada. \\
{*} El sistema calcula los valores a 16 cifras decimales. Se aproximan a dos
decimales.
\end{spacing}
}

\newpage

\section*{Metodología CBA}

Para obtener la Canasta Básica Alimentaria se aplicó la metodología propuesta
por CEPALPROGRESANSICA (2021), que representa varios cambios en la forma de
establecer los productos de dicha canasta, entre ellos se puede mencionar la
selección de la población de referencia, el uso de un deflactor geográfico, la
metodología de cálculo, etc.

\subsection{Definición del deflactor geográfico}

El deflactor geográfico es un índice utilizado para corregir posibles
disparidades de poder adquisitivo entre áreas geográficas dentro de un país,
asegurando que los hogares seleccionados tengan un nivel de vida comparable.
En Guatemala, dado que el Índice de Precios al Consumidor (IPC) se calcula sólo
a nivel urbano, se utiliza la Encuesta Nacional de Ingresos y Gastos de los
Hogares (ENIGH) 2022-2023 para su cálculo.
\\[\baselineskip]
Este proceso implicó la mensualización y expansión de los gastos diarios, la
agrupación de datos por hogar y producto, la generación de precios por producto
a nivel nacional, y la exclusión de productos extremos.
Luego, se generan precios medios y medianos por área geográfica, y se
identifican los alimentos comunes a zonas urbanas y rurales para ajustar el
gasto urbano.
El deflactor resultante fue de 0.925, indicando una diferencia de poder
adquisitivo entre áreas urbanas y rurales.

\subsection{Población de referencia}

Este método utiliza una forma de medir la pobreza llamada "Método de las
Necesidades Básicas Insatisfechas" (NBI).
Básicamente, lo que hace es evaluar si los hogares tienen carencias en ciertos
aspectos importantes de la vida, como servicios básicos (Fuentes de agua
mejorada, saneamiento mejorado), educación, alimentación (ingesta calórica,
participación del gasto en alimentos respecto del gasto total) o vivienda adecuada.
Si un hogar no cumple con al menos dos de estos aspectos, se considera como carente.
\\[\baselineskip]
Para seleccionar qué hogares formarán parte del estudio, primero se ordenan
según sus ingresos, es decir, cuánto dinero tiene.
Luego, se distribuyen en grupos de 20\% (quintiles móviles), donde el primer
grupo incluye a los hogares más pobres y el último a los más ricos.
Se selecciona el grupo donde por lo menos el 10\% de los hogares tiene al menos
dos carencias, lo que indica una situación de pobreza.

\newpage

\subsection{Selección de los productos}

De la población de referencia seleccionada se obtuvo una lista de los alimentos
que suelen consumir, a este listado se le llama lista larga, en la que se
determinó un total de 411 alimentos del área urbana y 386 del área rural
divididos en 14 grupos alimenticios.
\\[\baselineskip]
De esta lista inicial se obtuvieron dos listas cortas de alimentos que reflejan
los hábitos de consumo de la población en áreas urbanas y rurales, en donde se
seleccionaron aquellos que son más comunes y que son importantes para mantener
una dieta básica.
En esta lista se incluyeron los alimentos comprados fuera del hogar.
\\[\baselineskip]
La lista de alimentos incluye al menos dos opciones de cada grupo alimentario y
cada alimento está identificado claramente por su nombre. Se eligieron alimentos
que son consumidos por al menos el 10\% de los hogares, resultando en listas de
66 alimentos para el área urbana y 60 para el área rural.

\subsection{Cálculo de los costos de la CBA}

Para determinar el costo de la Canasta Básica de Alimentos (CBA), se utilizaron
los precios recopilados en el mes de \mes\ de \anio\ para el IPC base
diciembre de 2023.
Estos precios fueron sometidos a varios tratamientos estadísticos para eliminar
los valores atípicos.
Luego, se obtuvo el precio mediano, de acuerdo con la recomendación presentada
en la metodología actualizada por CEPAL en 2018 sobre la Medición de la Pobreza
por Ingresos.
Posteriormente, se aplicó la fórmula CMF (Costo Monetario de Adquisición de la
CBA por Fórmula en el mes), utilizada para calcular el costo total de adquirir
la CBA para un mes específico.
\\[\baselineskip]
En esta fórmula, cada producto individual que forma parte de la CBA tiene un
precio mediano observado (PMei), que refleja su costo en el mercado. Además, se
considera la cantidad diaria recomendada para el consumo de cada producto
(GFDi), medida en gramos, para satisfacer las necesidades nutricionales.
Para calcular el costo total de la CBA, se multiplica el precio mediano de cada
producto por la cantidad diaria recomendada, y luego se suman estos costos para
todos los productos de la CBA.
Por último, este total se multiplica por el número promedio de días en un mes
(30) para obtener el costo total estimado de adquirir la CBA durante ese
período.

\section*{Canasta Ampliada (CA)}

La Canasta Ampliada (CA) se define como el conjunto de bienes y servicios que
satisfacen las necesidades ampliadas de los miembros de un hogar y conforme los
datos declarados por los hogares, incluye alimentación, Bebidas alcohólicas,
ropa y calzado, vivienda, mobiliario, salud, transporte, comunicaciones,
recreación y cultura, educación, restaurantes y hoteles, servicios financieros y
cuidado personal. 

\section*{Canasta Ampliada Urbana (CAU)}

El cálculo de la CAU se obtiene por medio de un coeficiente, específicamente se
multiplica el costo per cápita mensual de la CBAU por el coeficiente de
Orshansky (2.421) con relación al gasto total en bienes y servicios, de acuerdo
con los resultados de la ENIGH 2022-2023.

{\color{AzulAcento} \[CCAU = CT\ CBAU \times 2.421\]}

Donde:

\begin{tabular}{r c l}
  CCAU & = & Costo de la Canasta Ampliada Urbana \\
  CT CCAU & = & Costo total mensual de la CBA Urbana
\end{tabular}
\\[\baselineskip]
Dado el costo per cápita mensual de la CBAU de Q.\CBAU, el costo de la CAU per
cápita mensual es de Q.\CAU\ a \mes\ del \anio.

\section*{Canasta Ampliada Rural (CAR)}

El cálculo de la CAR se obtiene por medio de un coeficiente, específicamente se
multiplica el costo per cápita mensual de la CBAR por el coeficiente de
Orshansky (1.968) con relación al gasto total en bienes y servicios, de acuerdo
con los resultados de la ENIGH 2022-2023.

{\color{AzulAcento} \[CCAR = CT\ CBAR \times 1.968\]}

Donde:

\begin{tabular}{r c l}
  CCAR & = & Costo de la Canasta Ampliada Rural \\
  CT CCAR & = & Costo total mensual de la CBA Rural
\end{tabular}
\\[\baselineskip]
Dado el costo per cápita mensual de la CBAU de Q.\CBAU, el costo de la CAU per
Dado el costo per cápita mensual de la CBAR de Q.\CBAR, el costo de la CAR per
cápita mensual es de Q.\CAR\ a \mes\ del \anio.

\includepdf[
  pagecommand={\thispagestyle{empty}}
]{./contraportada.pdf}

\end{document}
